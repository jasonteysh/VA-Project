\documentclass{acm_proc_article-sp}
\usepackage[utf8]{inputenc}

\renewcommand{\paragraph}[1]{\vskip 6pt\noindent\textbf{#1 }}
\usepackage{hyperref}
\usepackage{graphicx}
\usepackage{url}

\providecommand{\tightlist}{%
  \setlength{\itemsep}{0pt}\setlength{\parskip}{0pt}}

\title{User Guide - Understanding Airbnb listings in Australia}


% Add imagehandling

\numberofauthors{2}
\author{
\alignauthor Louelle Teo Fengmin \\
        \affaddr{Singapore Management University}\\
       \email{\href{mailto:louelle.teo.2020@mitb.smu.edu.sg}{\nolinkurl{louelle.teo.2020@mitb.smu.edu.sg}}}
\and \alignauthor Jason Tey Shou Heng \\
        \affaddr{Singapore Management University}\\
       \email{\href{mailto:jason.tey.2020@mitb.smu.edu.sg}{\nolinkurl{jason.tey.2020@mitb.smu.edu.sg}}}
\and \alignauthor Wong Kian Hoong \\
        \affaddr{Singapore Management University}\\
       \email{\href{mailto:kh.wong.2020@mitb.smu.edu.sg}{\nolinkurl{kh.wong.2020@mitb.smu.edu.sg}}}
\and }

\date{}

%Remove copyright shit
\permission{}
\conferenceinfo{} {}
\CopyrightYear{}
\crdata{}

% Pandoc syntax highlighting

% Pandoc citation processing
\newlength{\csllabelwidth}
\setlength{\csllabelwidth}{3em}
\newlength{\cslhangindent}
\setlength{\cslhangindent}{1.5em}
% for Pandoc 2.8 to 2.10.1
\newenvironment{cslreferences}%
  {}%
  {\par}
% For Pandoc 2.11+
\newenvironment{CSLReferences}[3] % #1 hanging-ident, #2 entry spacing
 {% don't indent paragraphs
  \setlength{\parindent}{0pt}
  % turn on hanging indent if param 1 is 1
  \ifodd #1 \everypar{\setlength{\hangindent}{\cslhangindent}}\ignorespaces\fi
  % set entry spacing
  \ifnum #2 > 0
  \setlength{\parskip}{#2\baselineskip}
  \fi
 }%
 {}
\usepackage{calc} % for calculating minipage widths
\newcommand{\CSLBlock}[1]{#1\hfill\break}
\newcommand{\CSLLeftMargin}[1]{\parbox[t]{\csllabelwidth}{#1}}
\newcommand{\CSLRightInline}[1]{\parbox[t]{\linewidth - \csllabelwidth}{#1}}
\newcommand{\CSLIndent}[1]{\hspace{\cslhangindent}#1}


\begin{document}
\maketitle


\emph{(The actual research paper itself should not more than 6 pages
excluding figures, tables, formula and references.)}

\hypertarget{introduction}{%
\section{Introduction}\label{introduction}}

Airbnb is an online marketplace platform for accomodation rental.
Founded in 2008 by Brian Chesky and Joe Gebbia who put an air mattress
in their living room and offered bed \& breakfast (thus ``Airbnb''), the
company has grown to be one of the most popular short-term accomodation
rental platforms in multiple countries around the world.

With millions of listings in 220 countries and over 100,000 cities,
Airbnb has a rich store of data from transactions between hosts and
guests. Such data includes structured data like price, number of
facilities (e.g.~bedrooms, bathrooms), minimum and maximum number of
nights' stay; and unstructured text data like description of the
accommodation, and reviews by guests.

This assignment focuses on analysis of unstructured text data from
Airbnb's online marketplace.

\hypertarget{motivation}{%
\section{Motivation}\label{motivation}}

\emph{(Motivation of the application)}

All of the studies discussed earlier focused on single dimensional
aspect of the dataset - for example, Kwon (2018) on predicting ratings,
Chen (2019) on visualizing the data, and Gedik (2020) on identifying
common amenities potentially useful in predicting demand and price. The
only attempt at merging the analyses into a centralised platform to be
used for thematic analysis is Gupta's. Notwithstanding the fact that his
Shiny apps are hosted on separate io pages, Gupta's analyses are also
disjointed in content. Gupta started off providing descriptive analysis
of Airbnb in New York, then transited to a Shiny for property locator,
then more exploratory analysis on geospatial and categortical
distribution, before transiting to temporal and text analysis -
throughout his study, there is no one common theme that links the
different methodologies and discussion together. A more concerted and
systematic organization of the analysis and the tools used is needed.

Across all the studies examined, including Gupta's Shiny app, there are
no provision of analytical tool that can allow users to change certain
parameters to tweak the analyses to their own objective/preference. For
example, in Kwon's predictive modeling, there is no way to change the
methods or predictor that is included, or perhaps even to change the
dependent variable of interest. In Chen's visualization, there is also
no way to change the variable presented, say from her original
price-as-size, to examine ratings using size or even colors on the
viz.~Similarly, Gedik does not offer any tools in this aspect. Gupta,
while offering a wide range of analytical methods and even offering
Shiny app, also surprisingly fell short in this area - for example, in
his text analytics app, while user can change the word query, they are
unable to, say, focus on reviews with more than a 4-star ratings to find
out common words associated with higher ratings.

Another gap that is found across all study and one of the most major
flaw considering the wealth of information provided by the datasets.
None of the study offered options to, for example, zoom in on a
particular sub-region, or to filter out whether a hosts is a superhost,
or to subset data based on a range of prices. These are useful features
that could tremendously enhance the usefulness of the analysis

\#Review of past works \emph{(Review and critique on past works)} The
data offered by Inside Airbnb are split into several different data sets
- this assignment will be focusing on the listings file, which provides
detailed data from individual listings put up on the AirBnb database.
Information provided includes hosts details, response data, listing
information (e.g.~price, number of beds, property type, maximum nights),
and review scores.

There are currently a number of analyses that has been conducted on the
Airbnb dataset. While most of them are focused on one particular
country, given that the datasets made available by Inside Airbnb are
rather consistent across different geographical space, it is reasonable
to assume that the analytical methodologies adopted by these other
studies can be easily replicated and reproduced with data from other
countries.

One of most prominent and recent study on Airbnb data is by Steve Deane
from Stratos, who wrote a blogpost in January 2021 on the topic (which
also inspired this project). In his post, Deane provides descriptive
statistics (e.g.~total number of hosts, most popular destination, number
of guests) focusing on the global distribution with a slight inclination
towards the United States of America. While some of the statistics were
provided, Deane's analysis is heavily weighted towards the economic
aspects of Airbnb (e.g.~whether Airbnb affects property values and rents
- ``Airbnb effect,'' whether Airbnb is cheaper than hotels, and economic
impact of Airbnb in each country). However, Deane does not provide any
higher level data analysis beyond the descriptive ones. He did presented
one segment explaining the factors that host should consider when
purchasing a property to use as an Airbnb - reflecting some form of
explanatory analysis (and perhaps even predictive) but the content was
scarce with no elaboration on how these factors were derived. The major
`flaw,' though, remains the fact that Deane's blog post is heavy on
qualitative write-up with minimal (or no) meaningful visualization of
the statistics he has quoted.

Several blog posts on medium.com provided basic guides on data analysis
on the Airbnb dataset. Kwon et. al.~(2018), Chen (2019), and Gedik
(2020) are some recent examples.

Kwon et. al.~(2018) used the Inside Airbnb listing data from Austin,
Texas with around 12,000 listings, and utilized the numerical features
to apply Linear Discriminant Analysis, Outliers were detected and
removed using the Cook's distance before a Box-Cox transformation was
conducted to normalize the data, and the dependent variable (review
score) binned to wrangle into categorical data type. Backward
Elimination, Ordinal Logistic Regression, and LASSO Regression methods
were employed to conduct explanatory analysis - with all three models
turning number of bathrooms as significant predictor for customer
ratings. Principal Component Analysis was then conducted with no good
result after the model fail to meaningfully provide separation of
classes (three classes). The Latent Discriminant Analysis was then
conducted on each of the three classes resulted from the PCA. The study
concluded that number of listings by hosts and having more bathrooms are
crucial in securing higher review score. The advantage of this study is
the depth, with detailed discussion on the methodology and results.
However, the analysis focuses only on predictive/explanatory model, and
is one-dimensional, studying only one dependent variable. The study does
not offer interactive features to allow other forms of data exploration
(e.g.~other clustering methods, geographical influence) or changing of
dependent variable (e.g.~to examine factors affecting price of listing).

Chen (2019) on the other hand provides an analysis that emphasized on
the geographical distribution of listings, and provided more data
visualizations (playing to the advantage of using Tableau) that allow
viewers to make quick noticeable trends. Chen analyzed the 2019 New York
City listing data with the objective to predict future Airbnb
performance in the city. However, while a predictive model was
envisioned, there was little predictive analysis going on in her study -
a greater emphasis was placed on qualitative inferences made via the
visualization without proper data analytics methodologies. Chen was also
limited in her methods of data visualization - for example, when
presenting the geospatial distribution of areas by price, she had used a
proportional dot map, without a convincing argument of its strength as
compared to other visualization methods (such as a choropleth). In fact,
the proportional size between the different dots are difficult to be
differentiated based on her viz.~There are also major flaws with other
visualization choice (e.g.~showing Average Price by Locations with
equal-length bar differentiated by colors on a continuous scale), and
the usefulness of some of the viz is also questionable (why might we be
interested to know who are the Top 10 Busy Host, and indeed, what is the
definition of `business' here). All in all, while Chen presented some
simple visualization to expose the potential of data viz using the
Airbnb data, more is left desired from the analysis.

Comparatively, Gedik (2020) did a fair job in answering the questions
set out by his objective. Using the Seattle listing dataset (via
Kaggle), he aimed to find out the common amenities, and top features
attracting guests and higher prices. In a Question-and-Answer format,
Gedik presented visualizations to help answer each of the question he
asked in a simple and succinct manner. In his third question, Gedik also
presented the result table of a linear regression model he had ran,
showing that property\_type\_Boat has the largest effect on price of
listing. Gedik's post, however, suffers from the limitation of scarce
discussion in the technical methodology aspect. Similar to earlier
studies, there is also no interactivity offered to allow viewers
autonomy in changing parameters.

Amongst all, Gupta (2019) provided the most well-rounded discussion and
presentation using the Airbnb listing data. Gupta aimed to provide an
exploratory analysis of Airbnb's data to understand the rental landscape
in New York City. Gupta first employed descriptive time-series
statistics to map out the increasing trends in number of listings and
reviews in the city, before moving on to present an interactive Shiny
App that provides information on individual listing based on sets of
filters (e.g.~max budget, number of people, minimum rating). While Gupta
offered some form of interactivity, the Shiny does not provide any
meaningful insights, and is essentially a replica of the user interface
offered by Airbnb via the official website.

While non-interactive, Gupta's subsequent discussion provides a
preliminary view of the analytical methodologies that could be employed
using the Inside Airbnb listing data. Firstly, he mapped out a
geographical spread of the ratings and price by area using a Choropleth
map (subsequent discussion in this paper will also conclude that it is
indeed one of the better geospatial visualization option for this
application). He also provided a quick bar-chart viz of the breakdown in
terms of property types, before moving on to discuss the temporal nature
and seasonality effect of price in New York City (using both a time
series dot plot to analyse monthly patterns and a boxplot for each day
of the week. He even employed a calendar heat map of occupancy within
the city. After geospatial and temporal analysis, Gupta moved on to
conduct text analytics on the reviews dataset to find words most
commonly mentioned - an attempt to distil aspects that play important
roles in shaping the Airbnb experience. A Shiny app was also developed
to allow user to find similar word vectors based on a query word.
Lastly, Gupta set out to find out whether there are any correlation
between the host response rate, the average ratings, and whether the
host is a Superhost. While no predictive model was used, the scatter
plot presented provides a quick viz on the explanatory analysis he
desired to conduct.

Within the industry, there exists interactive tools that allow analysts
or potential hosts to analyze rental data using attributes and past
performance. One such example is AIRDNA. Notwithstanding the fact that
the platform only offers paid services, the results are also provided in
a prescriptive manner with little analytical value-add.

\hypertarget{design-framework}{%
\section{Design Framework}\label{design-framework}}

\emph{(A detail description of the design principles used and data
visualisation elements built (Refer to Section IV: Interface of
\href{https://ink.library.smu.edu.sg/cgi/viewcontent.cgi?article=2760\&context=sis_research}{this
paper}.)}

\hypertarget{demonstration}{%
\section{Demonstration}\label{demonstration}}

\emph{(Use case)}

\#Discussion \emph{(What has the audience learned from your work? What
new insights or practices has your system enabled? A full blown user
study is not expected, but informal observations of use that help
evaluate your system are encouraged.)}

\#Future Work \emph{(A description of how your system could be extended
or refined.)}

\hypertarget{conclusion}{%
\section{Conclusion}\label{conclusion}}

Duis nec purus sed neque porttitor tincidunt vitae quis augue. Donec
porttitor aliquam ante, nec convallis nisl ornare eu. Morbi ut purus et
justo commodo dignissim et nec nisl. Donec imperdiet tellus dolor, vel
dignissim risus venenatis eu. Aliquam tempor imperdiet massa, nec
fermentum tellus sollicitudin vulputate. Integer posuere porttitor
pharetra. Praesent vehicula elementum diam a suscipit. Morbi viverra
velit eget placerat pellentesque. Nunc congue augue non nisi ultrices
tempor.

\hypertarget{references}{%
\section*{References}\label{references}}
\addcontentsline{toc}{section}{References}

\hypertarget{refs}{}
\begin{CSLReferences}{1}{0}
\leavevmode\hypertarget{ref-fenner2012a}{}%
Fenner, Martin. 2012. {``One-Click Science Marketing.''} \emph{Nature
Materials} 11 (4): 261--63. \url{https://doi.org/10.1038/nmat3283}.

\leavevmode\hypertarget{ref-meier2012}{}%
Meier, Reto. 2012. \emph{Professinal Android 4 Application Development}.
John Wiley \& Sons, Inc.

\end{CSLReferences}
\setlength{\parindent}{0in}

\end{document}
